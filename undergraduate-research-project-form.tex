%%%%%%%%%%%%%%%%%%%%%%%%%%%%%%%%%%%%%%%%%
% Sleek, Extensible Journal Article
% LaTeX Template
% Version 0.3
% Date 07/22/2014
%
% Original author:
% Calem Bendell
%
% License:
% CC BY-NC-SA 3.0 (http://creativecommons.org/licenses/by-nc-sa/3.0/)
%
%%%%%%%%%%%%%%%%%%%%%%%%%%%%%%%%%%%%%%%%%

\documentclass[10pt]{scrartcl}
\usepackage[T1]{fontenc}

\usepackage{lipsum}

\usepackage{kpfonts}
\usepackage{cabin}
\linespread{1.05}
\usepackage{microtype}

\usepackage[top=5em, left=4em, right=4em, bottom=7em, columnsep=2em]{geometry}
\usepackage[hang, small,labelfont=bf,up,textfont=it,up]{caption}
\usepackage[hidelinks, colorlinks=true]{hyperref}
\usepackage{booktabs, float, paralist, titlesec, abstract, titling, enumitem, graphicx, natbib, caption, subcaption, csvsimple, booktabs, multicol}
\usepackage[usenames,dvipsnames,svgnames,table]{xcolor}
\usepackage[flushmargin,hang,multiple]{footmisc}
\interfootnotelinepenalty=10000

\renewcommand{\abstractnamefont}{\normalfont\large\sffamily}
\renewcommand{\abstracttextfont}{\normalfont\itshape}

\titleformat*{\section}{\Large\sffamily}
\titleformat*{\subsection}{\large\sffamily}
\titleformat*{\subsubsection}{\itshape\sffamily}
\titleformat*{\paragraph}{\large\bfseries\sffamily}
\titleformat*{\subparagraph}{\large\bfseries\sffamily}

\setlist[description]{format=\normalfont\itshape}

\renewcommand{\maketitle}{\noindent\rule{\linewidth}{1pt}\LARGE \vspace{1em} \newline
\sffamily \scshape \thetitle \\
{ \normalsize \sffamily \theauthor }\\
\thedate
\newline\noindent\rule{\linewidth}{1pt}
\normalfont
\normalsize }


\def\DefaultHeightofCheckBox{0.8em}  
\def\DefaultWidthofCheckBox{0.8em}  
\def\DefaultHeightofText{1em}
\def\DefaultHeightofTextMultiline{4em}

% custom styling is necessary to give more freedom in how the title is displayed
% the existing style merely dictates the placement and rules above and below this section
\title{McGill Undergraduate 396/397 Research Project \vspace{.5em} \\
Project Title \vspace{.2em} \\
\TextField[name = projecttitle,
		   width = \textwidth, multiline = true, height = 2.2em, 
           value = Model Driven Development of GUI Applications and Their Generated Binding to Application Logic and the Domain Model,
           charsize = 15pt]
          {}}
\author{
McGill University \\
Fillable Form with Example Material and Course Description
}
\date{\vspace{-1em}}

\begin{document}

\begin{Form}
\maketitle

\section*{Course Description}

	Undergraduate research courses broaden the scope of a student's studies, improve availability of research opportunities to students, provide interdisciplinary researching opportunities, and opportune the learning and application of practical skills.
	
	The ideal undergraduate research project produces either a contribution to the researching community or an industry.
	These courses are open for every subject to all students in the Faculty of Science, e.g. a computer science student may take a BIOC 396 or MIMM 397.

\subsection*{Recommended Prerequisites}

	Should the student have a developed skillset that allows them to carry out their project, all prerequisites may be waived with approval of the supervising professor and the course administrator.
	
	Recommended prerequisites include:
	\begin{enumerate}
		\item At least one term of undergraduate study
		\item A CGPA of 3.0 or more
		\item Two or more 200-300 level courses in the chosen field
	\end{enumerate}

\subsection*{Course Terms}

		These courses are open to all students in the faculty of science, but cannot be taken more than once with the same professor.
		Additionally, each project must be for a different course.
		The courses are worth 3 credits.
		Your project is an elective that may be approved as a course towards a major with permission of both the supervising professor and the administrator for the course.
		
\subsection*{General Project Expectations}
		
		This section by no means specifies the expectations for \textit{your} undergraduate research project but is intended to provide a guideline for what may be expected of you for an appropriate project should you as the student want to propose your own.
		An undergraduate research project generally consists of a student doing original research in a field, often working towards a publication.
		Reproduction of significant results in the field have also been performed as excellent undergraduate projects.
		
		The final presentation, should there be one, will be approximately 10 minutes and will usually be delivered to the other students in your field also doing a 396/397 course.
		Accordingly, craft your presentation as though for a lecture hall to introduce your project and its purposes to a general undergraduate class.
		The details of the grading of your presentation should be discussed with your supervisor.
		
		The report is usually handed in before the last day of the semester, but sometimes will be requested earlier.
		Reports tend to be between 20 and 50 pages long; reports as long as 100 pages are neither unheard of nor likely read thoroughly; and reports are expected to be at least 10 good pages.
		You are highly encouraged to produce a professional report with high quality diagrams, proper figure legends, and multiple citations.
		To go above and beyond, try typesetting in LaTeX and preparing your report as though for publication.

\section{Project Information}

	\subsection{Project Description}
	
		\TextField[name = projectdescription, width = \textwidth, multiline = true, height = 10.2em, value = {Graphical user interface development can now typically be done through a WYSIWYG interface, allowing a programmer to develop an application through a code generating program up until the point of programming.
		Once a GUI is to be bound to application logic or a domain model, a programmer must return to code, usually binding an abstract event to clickable items.
		For more complicated applications, a GUI will repeatedly interact with application logic and update the view, usually through bindings.
		1. This project will create a modelling tool to gracefully bind a user view to an abstract domain model.
		2. A further goal of the project is to extend this modelling tool to bind parts of a user view directly to application logic.}]{}
		
	\subsection{Judging the Project}
	
		\TextField[name = projectjudgment, width = \textwidth, multiline = true, height = 4.2em, value = {The success of the project will be judged by the completion of the two clear goals listed in Section 1.1.
			The project will be deemed exceedingly well done if there is evidence that the goal was achieved elegantly and with ease of use in mind.}]{}

\begin{multicols}{2}
\section{Student Information}
	
	\sffamily
		\begin{tabular}{l|r}
			Name & \TextField[name = studentname, width = 13em, value = Calem Bendell]{} \\
			Student Number & \TextField[name = studentnumber, width = 13em, value = 260467886]{} \\
			Department & \TextField[name = studentdepartment, width = 13em, value = School of Computer Science]{} \\
			Year & \TextField[name = studentyear, width = 13em, value = U3]{} \\
			\bottomrule
		\end{tabular}
	\normalfont

\section{Grading}

	\sffamily
	\begin{tabular}{l|r}
			 Description & Grade Percentage \\
		\midrule \\
			End of Semester Presentation & \TextField[name = presentationpercent, width = 2em, value = 10]{} \\
			Final Report	& \TextField[name = reportpercent, width = 2em, value = 60]{} \\
			Laboratory Performance	& \TextField[name = laboratorypercent, width = 2em, value = 30]{} \\
		\bottomrule 
	\end{tabular}
\end{multicols}

\vfill

\section{Signatures}

\subsection{Student}

\TextField[name = studentsig, width=20em, height=1.5em, value = Calem Bendell]{}

\subsection{Supervisor}

\TextField[name = supervisorsig, width=20em, height=1.5em, value = J\"org Kienzle]{}

\subsection{Course Administrator}

\TextField[name = administratorsig, width=20em, height=1.5em, value = Nathan Friedman]{}

\vfill
\color{lightgray} \centering for questions or concerns regarding this form \\ visit https://github.com/mcgill-csus/undergraduate-research-project-form/ \\ or email calem.bendell@mail.mcgill.ca or victor.chisholm@mcgill.ca

\end{Form}
\end{document}